\documentclass[11pt]{article}
\usepackage{multirow}
\usepackage{hyperref}
\usepackage[margin=0.75in, top=1in]{geometry} 
\usepackage{titling} 
\usepackage{fancyhdr}
\usepackage{cancel}
\usepackage{amsmath}
\usepackage{graphicx}
\usepackage{ragged2e}
\usepackage{calc}
\usepackage{here}
\usepackage{siunitx}




\setlength{\droptitle}{-3cm} 
\pagestyle{fancy}
\fancyhf{} 
\lhead{EN.566: Introduction to Computational Materials Modeling 2023 - HW3}
\rhead{Amitabh Roy}

\begin{document}

\section{Problem Description}
The objective of the HW3 is to solve the problem of Oscillatory Motion and Chaos and also to solve Poisson equation. The Oscillatory problem can be solved using Euler-Cromer or Runge-Kutta method, And to solve the Poisson equation we can use Jacobi or gauss-seidel. Detailed explanation can be found below:

\section{Solution to Pendulum Problem}

\subsection{Calculate analytically at what (approximate) value of $\Omega_D$ the resonance occurs. Do you expect the small-angle (linear) approximation to be good?}

\begin{equation}
\frac{d^2\theta}{dt^2} = -g\theta - 2\gamma\frac{d\theta}{dt} + \alpha_D\sin(\Omega_D t) \tag{1}
\end{equation}

\begin{equation}
\theta'' + 2\gamma\theta' + \omega_0^2\theta = F(t) \quad F(t) = \alpha_D \sin(\Omega_D t) \tag{2}
\end{equation}

Assuming $F(t) = 0$, such that $\theta(t) \sim e^{rt}$, where $r$ is a root.

\begin{equation}
r^2 + 2\gamma r + \omega_0^2 = 0 \quad r_\pm = -\gamma \pm \sqrt{\gamma^2 - \omega_0^2} \tag{3}
\end{equation}

\begin{equation}
\theta(t)_{\text{Homogeneous}} = 
(Ae^{rt} + Be^{-rt})e^{-\gamma t} \quad \text{if } \gamma \neq \omega_0 \tag{5.1}
\end{equation}

\begin{equation}
\theta(t)_{\text{Homogeneous}} = (A + Bt)e^{-\gamma t} \quad \text{if } \gamma = \omega_0
\tag{5.2}
\end{equation}

Condition: steady-state (ss)

\begin{equation}
\theta(t)_{\text{ss}} = \theta_P\sin(\Omega_D t - \phi) \tag{6}
\end{equation}

Puting eqn (6) into (2) with $F(t) = \alpha D\sin(\Omega_D t)$ results in:

\begin{equation}
\alpha_D\sin(\Omega_D t) = \frac{\omega_0^2 - \Omega_D^2}{\alpha}\sin(\Omega_D t - \phi) + 2\gamma\Omega_D\cos(\Omega_D t - \phi) \tag{7}
\end{equation}

As,\begin{equation}
\sin(\alpha + \beta) = \sin(\alpha)\cos(\beta) + \cos(\alpha)\sin(\beta) \tag{8}
\end{equation}

\begin{equation}
\sin(\alpha + \beta) = \sin(\alpha)\left(\frac{\theta_P}{\alpha_D}(\omega_0^2 - \Omega_D^2)\right) + \cos(\alpha)\left(\frac{2\gamma\Omega_D\theta_P}{\alpha_D}\right) \tag{9}
\end{equation}

Therefore:

\begin{equation}
\cos(\phi) = \frac{\theta_P}{\alpha_D}(\omega_0^2 - \Omega_D^2)
\tag{10}
\end{equation}

\begin{equation}
\sin(\phi) = \frac{2\gamma\Omega_D\theta_P}{\alpha_D}
\tag{11}
\end{equation}


Finding, $\theta_P(\Omega_D)$ and $\phi(\Omega_D)$  from $\sin(\phi)$ and $\cos(\phi)$


\begin{equation}
\theta_P(\Omega_D) = \frac{\alpha_D}{\sqrt{(\omega_0^2 - \Omega_D^2)^2 + 4\gamma^2\Omega_D^2}} \tag{12}
\end{equation}


\begin{equation}
\phi(\Omega_D) = \arctan\left(\frac{2\gamma\Omega_D}{\omega_0^2 - \Omega_D^2}\right) \tag{13}
\end{equation}

To find $\Omega_{\text{Res}}$ analytically, we can equate the first derivative of to zero:

\begin{equation}
\frac{d\theta_P}{d\Omega} = 0  \tag{13}
\end{equation}

On solving we will get

\begin{equation}
    \quad \Omega_{\text{Res}} = \sqrt{\omega_0^2 - 2\gamma^2} \tag{14}
\end{equation}

On taking,
\begin{equation}
    \omega_0 = 1 {sec^{-1}} \tag{15}
\end{equation}

\begin{equation}
    \gamma = 0.25 {sec^{-1}} \tag{16}
\end{equation}

We get,
\begin{equation}
    \Omega_{\text{Res}} = 0.935 \text{ rad/sec} \tag{17}
\end{equation}

\end{document}