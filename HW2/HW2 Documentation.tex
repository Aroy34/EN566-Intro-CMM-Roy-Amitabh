\documentclass[12pt]{article}
\usepackage{multirow}
\usepackage{hyperref}
\usepackage[margin=1in, top=1.5in]{geometry} 
\usepackage{titling} 
\usepackage{fancyhdr}


\setlength{\droptitle}{-3cm} 
\pagestyle{fancy}
\fancyhf{} 
\lhead{EN.566: Introduction to Computational Materials Modeling 2023 - HW2}
\rhead{Amitabh Roy}

\begin{document}

\section{Problem Description}
The objective of the HW2 is to solve the problem of radioactive decay of C-14 isotope and also to predict the trajectory of the gold ball. The radioactivity and the golf trajectory problem can be solved using the Euler's method and finally write a python code for doing the calculations.  Mathematical relations for section can be found below:

\section{Solution to Carbon Dating Problem}

In this problem we have to derive a relation between half-life time $T_{1/2}$ and decay constant $\tau$,  plot the number of particles left vs time over the duration of 20,000 years and with different step size (10, 100, 1000 years)

\subsection{Derive analytically the relation between half-life time $T_{1/2}$ and decay constant $\tau$ as defined in class.}
let,\\ 
N = Number of nuclie at any given time 't'\\
t = time\\
$\Delta$N = Number of nuclie undergoing decay at time 't'\\
So, $\Delta$N $\propto $ N*$\Delta$t\\
$\Delta$N = -$\tau$ N*$\Delta$t {Where, $\tau$ is the decay constant}\\
On rearranging
\frac{ $\Delta$N}{$\Delta$t}



\begin{figure}[h]

\end{figure}

\subsection{Subtopic 1.2}
Your solution and explanation for subproblem 1.2.

\section{Solution to Problem 2}
\subsection{Subtopic 2.1}
Your solution and explanation for subproblem 2.1.

\begin{figure}[h]
 
\end{figure}

\subsection{Subtopic 2.2}
Your solution and explanation for subproblem 2.2.

\section{Contributions}
In this section, you can highlight the contributions of your work, such as novel approaches, insights, or findings.

  
\end{document}